% Definición del documeto
\documentclass[11pt,letterpaper]{article}
\usepackage[utf8]{inputenc}
\usepackage[spanish]{babel}
% Margenes del documeto
\usepackage[left=2cm,right=2cm,top=1.8cm,bottom=2.3cm]{geometry}
\usepackage{fancyhdr}
% Imágenes
\usepackage{epsfig,graphicx}
\usepackage[demo]{graphicx}
\usepackage{subfig}
% Matemáticas
\usepackage{enumitem}
\usepackage{upgreek}
\usepackage{amsmath}
\usepackage{amssymb}
\usepackage{amsthm}
% Referencias 
\usepackage{hyperref}
\usepackage{url}
% Código
\usepackage{minted}
\usemintedstyle{borland}
\usepackage{mdframed}
\setminted[python]{breaklines}
% Colores y figuras
\usepackage{xcolor}
\usepackage{tcolorbox,tikzlings}
\tcbuselibrary{skins,xparse,listings}
\usepackage{color, colortbl}
% Secciones
\usepackage{titlesec}
\titlelabel{\thetitle.\quad}
% Modificar comandos
\usepackage{multicol, array}
\usepackage{xpatch, calc}

% ================ Ubuntu Terminal ================ %


\newtcblisting{ubuntu}{colback=blue!30!black,
colupper=white,colframe=gray!65!black,listing only,
listing options={style=tcblatex,language=sh,escapeinside=``,},
title={\textcolor{red}{\Huge{$\bullet$}}{\textcolor{gray}{\Huge{$\bullet\bullet$}}}},
every listing line={\MyUbuntuPrompt}}
\pgfkeys{/ubuntu/.cd,
user/.code={\gdef\MyUbuntuUser{#1}},user={},
host/.code={\gdef\MyUbuntuHost{#1}},host={},
color/.code={\gdef\MyUbuntuColor{#1}},color=red,
prompt char/.code={\gdef\MyUbuntuPromptChar{#1}},prompt char= \,\,,
root/.style={user=root,host=unam,color=lime,prompt char=\$},
fac/.style={user={unam},host=Practica1,color=lime,prompt char=\$},
}
\newcommand{\SU}[1]{\pgfkeys{/ubuntu/.cd,#1}%
\gdef\MyUbuntuPrompt{\textcolor{\MyUbuntuColor}{\small\ttfamily\bfseries \textcolor{cyan}{\MyUbuntuUser}\textcolor{white}{@}\MyUbuntuHost{\textcolor{white}>}\textcolor{lime}{\url{}}{\textcolor{white}\MyUbuntuPromptChar}}}}
\newcommand{\StartConsole}{\gdef\MyUbuntuPrompt{}}
\SU{user=fciencias,host=dha,color=cyan}

\newcommand{\quotebox}[1]{\begin{center}\fcolorbox{white}{blue!15!gray!15}{\begin{minipage}{0.9\linewidth}\vspace{10pt}\center\begin{minipage}{0.8\linewidth}{\space\Huge``}{#1}{\hspace{1.5em}\break\null\Huge\hfill''}\end{minipage}\smallbreak\end{minipage}}\end{center}}

% ================ Colored Quotation ================ %
\newsavebox{\coloredquotationbox}
\newenvironment{coloredquotation}
 {%
  \begin{trivlist}
  \begin{lrbox}{\coloredquotationbox}
  \begin{minipage}{\dimexpr\linewidth-2\fboxsep}
 }
 {%
  \end{minipage}
  \end{lrbox}
  \item\relax
  \parbox{\linewidth}{
    \begingroup
    \color[RGB]{16,38,148}%
    \hrule
    \color[RGB]{16,38,148}%
    \hrule
    \color[RGB]{16,38,148}%
    \hrule
    \endgroup
    \colorbox[RGB]{199,227,253}{\usebox{\coloredquotationbox}}\par\nointerlineskip % Cambiar el color aquí
    \begingroup
    \color[RGB]{16,38,148}%
    \hrule
    \color[RGB]{16,38,148}%
    \hrule
    \color[RGB]{16,38,148}%
    \hrule
    \endgroup
  }
  \end{trivlist}
 }

% Columnas ara el header
\newcolumntype{P}[1]{>{\centering\arraybackslash}p{#1}}
\newcolumntype{M}[1]{>{\centering\arraybackslash}m{#1}}
\newcolumntype{R}[1]{>{\arraybackslash}m{#1}}

% ================ Header ================ %
 \newcommand{\thesubject}%
	{Fundamentos de Bases de Datos}

\newcommand{\firstinstitute}%
	{Facultad de Ciencias}
	
\newcommand{\shortinstitute}%
	{Facultad de Ciencias, UNAM}

\pagestyle{fancy}
\fancyhf{}
\fancyhead[l]{%
    \begin{tabular}
        {@{}R{2.25em}@{}R{\widthof{\shortinstitute\ }}}
        \includegraphics[height=.25in]{logos/fcunam.png} & \shortinstitute
    \end{tabular}
}
\fancyhead[r]{%
    \textit{\thesubject}
}

\fancyfoot[c]{%
    \begin{tabular}
        {M{1em}@{}M{2em}@{}M{1em}}
	\textcolor{black}{$\boldsymbol{-}$} & \thepage &
        \textcolor{black}{$\boldsymbol{-}$}
    \end{tabular}
}
