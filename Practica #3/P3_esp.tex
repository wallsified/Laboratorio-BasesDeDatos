% Definición del documeto
\documentclass[11pt,letterpaper]{article}
\usepackage[utf8]{inputenc}
\usepackage[spanish]{babel}
% Margenes del documeto
\usepackage[left=2cm,right=2cm,top=1.8cm,bottom=2.3cm]{geometry}
\usepackage{fancyhdr}
% Imágenes
\usepackage{epsfig,graphicx}
\usepackage[demo]{graphicx}
\usepackage{subfig}
% Matemáticas
\usepackage{enumitem}
\usepackage{upgreek}
\usepackage{amsmath}
\usepackage{amssymb}
\usepackage{amsthm}
% Referencias 
\usepackage{hyperref}
\usepackage{url}
% Código
\usepackage{minted}
\usemintedstyle{borland}
\usepackage{mdframed}
\setminted[python]{breaklines}
% Colores y figuras
\usepackage{xcolor}
\usepackage{tcolorbox,tikzlings}
\tcbuselibrary{skins,xparse,listings}
\usepackage{color, colortbl}
% Secciones
\usepackage{titlesec}
\titlelabel{\thetitle.\quad}
% Modificar comandos
\usepackage{multicol, array}
\usepackage{xpatch, calc}

% ================ Ubuntu Terminal ================ %


\newtcblisting{ubuntu}{colback=blue!30!black,
colupper=white,colframe=gray!65!black,listing only,
listing options={style=tcblatex,language=sh,escapeinside=``,},
title={\textcolor{red}{\Huge{$\bullet$}}{\textcolor{gray}{\Huge{$\bullet\bullet$}}}},
every listing line={\MyUbuntuPrompt}}
\pgfkeys{/ubuntu/.cd,
user/.code={\gdef\MyUbuntuUser{#1}},user={},
host/.code={\gdef\MyUbuntuHost{#1}},host={},
color/.code={\gdef\MyUbuntuColor{#1}},color=red,
prompt char/.code={\gdef\MyUbuntuPromptChar{#1}},prompt char= \,\,,
root/.style={user=root,host=unam,color=lime,prompt char=\$},
fac/.style={user={unam},host=Practica1,color=lime,prompt char=\$},
}
\newcommand{\SU}[1]{\pgfkeys{/ubuntu/.cd,#1}%
\gdef\MyUbuntuPrompt{\textcolor{\MyUbuntuColor}{\small\ttfamily\bfseries \textcolor{cyan}{\MyUbuntuUser}\textcolor{white}{@}\MyUbuntuHost{\textcolor{white}>}\textcolor{lime}{\url{}}{\textcolor{white}\MyUbuntuPromptChar}}}}
\newcommand{\StartConsole}{\gdef\MyUbuntuPrompt{}}
\SU{user=fciencias,host=dha,color=cyan}

\newcommand{\quotebox}[1]{\begin{center}\fcolorbox{white}{blue!15!gray!15}{\begin{minipage}{0.9\linewidth}\vspace{10pt}\center\begin{minipage}{0.8\linewidth}{\space\Huge``}{#1}{\hspace{1.5em}\break\null\Huge\hfill''}\end{minipage}\smallbreak\end{minipage}}\end{center}}

% ================ Colored Quotation ================ %
\newsavebox{\coloredquotationbox}
\newenvironment{coloredquotation}
 {%
  \begin{trivlist}
  \begin{lrbox}{\coloredquotationbox}
  \begin{minipage}{\dimexpr\linewidth-2\fboxsep}
 }
 {%
  \end{minipage}
  \end{lrbox}
  \item\relax
  \parbox{\linewidth}{
    \begingroup
    \color[RGB]{16,38,148}%
    \hrule
    \color[RGB]{16,38,148}%
    \hrule
    \color[RGB]{16,38,148}%
    \hrule
    \endgroup
    \colorbox[RGB]{199,227,253}{\usebox{\coloredquotationbox}}\par\nointerlineskip % Cambiar el color aquí
    \begingroup
    \color[RGB]{16,38,148}%
    \hrule
    \color[RGB]{16,38,148}%
    \hrule
    \color[RGB]{16,38,148}%
    \hrule
    \endgroup
  }
  \end{trivlist}
 }

% Columnas ara el header
\newcolumntype{P}[1]{>{\centering\arraybackslash}p{#1}}
\newcolumntype{M}[1]{>{\centering\arraybackslash}m{#1}}
\newcolumntype{R}[1]{>{\arraybackslash}m{#1}}

% ================ Header ================ %
 \newcommand{\thesubject}%
	{Fundamentos de Bases de Datos}

\newcommand{\firstinstitute}%
	{Facultad de Ciencias}
	
\newcommand{\shortinstitute}%
	{Facultad de Ciencias, UNAM}

\pagestyle{fancy}
\fancyhf{}
\fancyhead[l]{%
    \begin{tabular}
        {@{}R{2.25em}@{}R{\widthof{\shortinstitute\ }}}
        \includegraphics[height=.25in]{logos/fcunam.png} & \shortinstitute
    \end{tabular}
}
\fancyhead[r]{%
    \textit{\thesubject}
}

\fancyfoot[c]{%
    \begin{tabular}
        {M{1em}@{}M{2em}@{}M{1em}}
	\textcolor{black}{$\boldsymbol{-}$} & \thepage &
        \textcolor{black}{$\boldsymbol{-}$}
    \end{tabular}
}

\begin{document}
\thispagestyle{empty}
\rule{17cm}{0.1mm}
\begin{center}
    \begin{minipage}{3cm}
        \begin{center}
    	\includegraphics[height=3.5cm]{logos/Logo_UNAM.png}
        \end{center}
    \end{minipage}\hfill
    \begin{minipage}{10cm}
        \begin{center}
            \textbf{\large Universidad Nacional Autónoma de México}\\[0.1cm]
            \textbf{\large Facultad de Ciencias}\\[0.2cm]          
            \Large \textbf{Fundamentos de Bases de Datos}\\[0.2cm]
            \LARGE \textbf{Práctica 3}  \\[0.1cm]
            \Large \textbf{Creación de Tablas Complejas y Consultas Avanzadas}  \\[0.1cm]
            \vspace{0.2cm}
            \small 
                Autor: Arroyo Martínez Erick Daniel  \\[0.1cm] 
        \end{center}
    \end{minipage}\hfill
    \begin{minipage}{3cm}
        \begin{center}
            \includegraphics[height=3.5cm]{logos/Logo_FC.png}
        \end{center}
    \end{minipage}
\end{center}

\rule{17cm}{0.1mm}

\section*{Introducción}

En esta práctica, los estudiantes trabajarán con un esquema de base de datos complejo previamente implementado. El objetivo es ejecutar y analizar una serie de consultas SQL que emplean técnicas avanzadas. Estas consultas abordarán una variedad de temas, desde la selección y filtrado de datos hasta la agregación y uso de funciones avanzadas.

\section*{Objetivos}

\begin{enumerate}
\item Implementar un esquema \ref{fig:diagramaP3_1} de base de datos complejo con múltiples tablas y relaciones en Oracle LiveSQL.
\item Realizar consultas SQL avanzadas para extraer y analizar datos del esquema complejo.
\item Aplicar una variedad de técnicas SQL, incluyendo agregaciones, condiciones de filtrado, y manipulación de datos.
\item Comprender y utilizar correctamente funciones y operadores SQL avanzados.
\end{enumerate}

\section*{Especificaciones de Desarrollo}

\subsection*{Paso 1: Implementación del Esquema}

\begin{enumerate}
    \item Crea las siguientes tablas en Oracle LiveSQL según el diagrama relacional proporcionado:
    \begin{itemize}
        \item \textbf{Departments}
        \item \textbf{Employees}
        \item \textbf{Projects}
        \item \textbf{Project\_Assignments}
        \item \textbf{Salaries}
    \end{itemize}
    \item Asegúrate de incluir las restricciones de integridad correspondientes para cada tabla, como \texttt{NOT NULL}, \texttt{AUTO\_INCREMENT}, \texttt{UNIQUE}, y \texttt{CHECK}.
\end{enumerate}
%% Diagrama
\begin{figure}[h]
    \centering
    \includegraphics[width=0.85\linewidth]{Images/Diagrama_P3_0.png}
    \caption{Diagrama Relacional}
    \label{fig:diagramaP3_1}
\end{figure}


\subsection*{Paso 2: Inserción de Datos}

\begin{enumerate}
    \item Inserta datos de ejemplo en cada tabla. Asegúrate de que los datos respeten las restricciones de integridad establecidas.
    \item Los datos deben ser suficientes para poder realizar las consultas básicas requeridas posteriormente.
\end{enumerate}

\subsection*{Paso 3: Consultas Avanzadas}
\begin{tcolorbox}[colback=red!5!white, colframe=red!75!black, title= Restricción]
\begin{verbatim}
    No deben usar Joins
\end{verbatim}
\end{tcolorbox}

\begin{enumerate}

    \item Selecciona los nombres y apellidos de todos los empleados que trabajan en un departamento con un presupuesto mayor a 1,000,000.
    \\
    \textbf{Hint:} Usa una subconsulta para filtrar los departamentos con un presupuesto superior a un millón.
    \begin{verbatim}
SELECT fst_name, lst_name FROM Employees 
WHERE department_id IN (SELECT department_id FROM Departments WHERE budget > 1000000);
    \end{verbatim}

    \item Obtén los nombres y apellidos de los empleados que tienen un salario mayor al promedio de todos los salarios. Ordena los resultados por el salario en orden descendente.
    \\
    \textbf{Hint:} Utiliza la función de agregación que calcula el promedio para comparar los salarios individuales con el promedio.
    \begin{verbatim}
SELECT fst_name, lst_name FROM Employees 
WHERE salary > (SELECT ROUND(AVG(salary)) from Employees);
    \end{verbatim}

    \item Encuentra todos los proyectos que tienen un presupuesto entre 500,000 y 1,000,000. Muestra el nombre del proyecto y su presupuesto.
    \\
        \textbf{Hint:} Usa la cláusula para filtrar los presupuestos dentro del rango especificado.
    \begin{verbatim}
SELECT project_name, budget from Projects WHERE budget BETWEEN 500000 AND 1000000;
    \end{verbatim}

    \item Inserta un nuevo empleado en la tabla \texttt{Employees} con un salario y asigna este empleado a un departamento específico.
    \\
    \textbf{Hint:} Asegúrate de asignar correctamente el ID del departamento en la nueva fila que insertes.
    \begin{verbatim}
INSERT INTO Employees (employee_id, fst_name, lst_name, email, phone_number, hire_date, job_title, department_id, salary) 
    VALUES (11, 'Daniel', 'Paredes', 'danielparedes@ciencias.unam.mx', '556-937-7986', TO_DATE('2024-08-22', 'YYYY-MM-DD'), 'IT Manager', 3, 8500);
    \end{verbatim}

    \item Actualiza el salario de todos los empleados cuyo título de trabajo es 'Manager' incrementándolo en un 10\%.
    \\
    \textbf{Hint:} Usa una operación matemática para incrementar los salarios en función de su valor actual.
    \begin{verbatim}
UPDATE Employees SET salary = (salary + (salary*.10)) WHERE job_title LIKE '%Manager';
    \end{verbatim}

    \item Elimina todos los registros de empleados que no tienen ningún proyecto asignado.
    \\
    \textbf{Hint:} Utiliza una subconsulta o un operador de comparación para identificar empleados sin asignaciones de proyectos.
    \begin{verbatim}
DELETE FROM Employees WHERE employee_id NOT IN (SELECT employee_id FROM Project_Assignments);
    \end{verbatim}

    \item Cuenta cuántos empleados trabajan en cada departamento y muestra el nombre del departamento junto con el número de empleados. Filtra los resultados para mostrar solo los departamentos con más de 10 empleados.
    \\
    \textbf{Hint:} Usa la función de agregación que cuenta filas para agrupar los empleados por departamento.
    \begin{verbatim}
SELECT d.department_name, employee_count
FROM Departments d,
     (SELECT department_id, COUNT(*) as employee_count
      FROM Employees
      GROUP BY department_id
      HAVING COUNT(*) > 10) e
WHERE d.department_id = e.department_id;
    \end{verbatim}

    \item Encuentra el salario mínimo y máximo que reciben los empleados en el departamento de 'IT'. Muestra el título del trabajo, el salario mínimo y el salario máximo.
    \\
    \textbf{Hint:} Usa funciones de agregación para calcular los valores mínimo y máximo.
    \begin{verbatim}
SELECT MIN(salary) AS ITSalaryMin, MAX(salary) AS ITSalaryMax FROM Employees WHERE department_ID = 3;
    \end{verbatim}

    \item Encuentra los nombres de los proyectos que comenzaron en el año 2023 y que están en el departamento con el presupuesto más alto.
    \\
    \textbf{Hint:} Usa una subconsulta para obtener el presupuesto más alto y filtra los proyectos por la fecha de inicio.
    \begin{verbatim}
SELECT DISTINCT p.project_name
FROM Projects p
WHERE p.project_id IN (
    SELECT pa.project_id
    FROM Project_Assignments pa
    WHERE pa.employee_id IN (
        SELECT e.employee_id
        FROM Employees e
        WHERE e.department_id = (
            SELECT department_id
            FROM Departments
            WHERE budget = (SELECT MAX(budget) FROM Departments)
        )
    )
)
AND EXTRACT(YEAR FROM p.start_date) = 2023;
    \end{verbatim}

    \item Selecciona los nombres y apellidos de los empleados que tienen un correo electrónico que contiene el dominio 'example.com'. Asegúrate de que el correo electrónico sea único.
    \\
    \textbf{Hint:} Filtra los resultados usando un patrón que coincida con el dominio 'example.com' en la dirección de correo electrónico.
    \begin{verbatim}
SELECT DISTINCT fst_name, lst_name
FROM Employees
WHERE email LIKE '%@example.com';
    \end{verbatim}

    \item Muestra los nombres de los empleados que están trabajando en más de un proyecto. Ordena los resultados alfabéticamente.
    \\
    \textbf{Hint:} Agrupa los resultados por empleado y filtra aquellos que tienen más de una asignación de proyecto.
    \begin{verbatim}
SELECT e.fst_name, e.lst_name
FROM Employees e
WHERE e.employee_id IN (
    SELECT employee_id
    FROM Project_Assignments
    GROUP BY employee_id
    HAVING COUNT(*) > 1
)
ORDER BY e.fst_name, e.lst_name;
    \end{verbatim}

    \item Selecciona el nombre del proyecto y el total de horas asignadas a cada proyecto. Filtra los resultados para mostrar solo los proyectos con más de 1000 horas asignadas.
    \\
    \textbf{Hint:} Usa una función de agregación para sumar las horas asignadas y filtra los proyectos que cumplan con el criterio.
    \begin{verbatim}
SELECT p.project_name, total_hours
FROM Projects p,
     (SELECT project_id, SUM(hours_allocated) as total_hours
      FROM Project_Assignments
      GROUP BY project_id
      HAVING SUM(hours_allocated) > 1000) pa
WHERE p.project_id = pa.project_id;
    \end{verbatim}

\end{enumerate}

\section*{Justificación}

En esta sección, deben explicar las relaciones entre las diferentes tablas del esquema y la cardinalidad de cada relación. 
Deben describir cómo estas relaciones y la cardinalidad asociada impactan en el diseño del esquema y la integridad de los datos. Además, deben explicar cómo las restricciones de integridad, como las claves foráneas y las verificaciones, garantizan la consistencia y coherencia de los datos en todo el sistema.
\begin{itemize}
    \item \textbf{Departments \textit{y} Employees:} 
    \item \textbf{Departments \textit{y} Projects:} 
    \item \textbf{Employees \textit{y} Project\_Assignments:} 
    \item \textbf{Projects \textit{y} Project\_Assignments:} 
    \item \textbf{Employees \textit{y} Salaries:} 
\end{itemize}
\section*{Entregables}

\begin{itemize}
    \item Un archivo \texttt{.sql} que contenga la implementación completa del esquema de base de datos, incluyendo la creación de tablas y la inserción de datos.
    \item Capturas de pantalla que evidencien:
    \begin{itemize}
        \item La creación de cada tabla con sus restricciones.
        \item Las consultas realizadas y los resultados obtenidos.
    \end{itemize}
    \item Un archivo PDF con las capturas de pantalla y una breve explicación de cada paso realizado.
\end{itemize}

\section*{Rúbrica de Evaluación}

\begin{tabular}{|l|l|c|}
\hline
\textbf{Criterio} & \textbf{Descripción} & \textbf{Ponderación} \\
\hline
\textbf{Creación de tablas} & Implementación correcta de las tablas con restricciones de integridad & 15\% \\
\hline
\textbf{Inserción de datos} & Inserción correcta y coherente de datos en las tablas & 10\% \\
\hline
\textbf{Consultas básicas} & Realización correcta de las consultas y resultados obtenidos & 50\% \\
\hline
\textbf{Justificación} & Claridad y corrección de argumentos & 15\% \\
\hline
\textbf{Presentación} & Claridad y organización de las capturas de pantalla y explicaciones & 10\% \\
\hline
\end{tabular}

\input{desarrollo/Recursos}
\end{document}